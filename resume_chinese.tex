\documentclass[UTF8]{resume}

\name{邱江坤}
\address{Tel: +86 158 6659 6952 • Email: \href{mailto://qiujiangkun@foxmail.com}{qiujiangkun@foxmail.com} • \href{https://github.com/qiujiangkun/}{GitHub: qiujiangkun}}
% TODO: 添加小图标

\address{实习目标: 软件开发工程师 后端方向}

\begin{document}
\begin{rSection}{自我评价}
    \textbf{编程经验}:9年编程经验,热衷开源,能够独自或合作编写上万行代码的项目\\
    \textbf{编程语言}:熟悉C++,Java,Python,Rust,作为开发新项目的主要语言;了解JavaScript, Bash, Scala, SQL, C\#,PHP,汇编\\
    \textbf{技术栈}:熟悉Linux、git、计算机底层、机器学习、深度学习;了解POSIX、Netty、Disruptor、并发编程、JVM调优、编译原理\\
\end{rSection}

\begin{rSection}{教育经历}
    \begin{itemize}
        \item 香港科技大学~综合系统与设计(编程方向)专业\hfill 2020.9-今
    \end{itemize}
\end{rSection}

\begin{rSection}{实习经历}
    \begin{rExperience}{软件开发工程师}{数字货币交易公司(保密)}{2020.07-今}
        \textbf{项目背景}\\
        升级现有客户端并发网络库,降低交易网站信息收集处理数据延迟\\
        \textbf{项目内容}
        \begin{itemize}
            \itemsep -0.5em \vspace{-0.5em}
            \item 搭建网络库:负责高频交易的底层库从0到1的搭建;使用Rust语言,运用缓存行优化和无锁队列等底层优化,实现了一套处理客户端并发的网络库,并支持多种通信协议;通过修改Webpki和Rustls等库,支持对IP地址的TLS的链接;以极低的延迟从不同的交易平台收集并处理数据;正在构建基于f-stack的用户态网络库;基于UDP实现自定义协议,数据中转程序,绕开服务器机房对于IP的限制
            \item搭建监控系统:运用InfluxdDB和Grafana搭建监控系统,实时监控不同交易平台数据采集处理情况
            \item搭建绘图引擎:修改Matplotlib,支持大量数据的实时绘图,同时兼容Matplotlib所有的特性
            \item项目成果:总延迟与Apache Benchmark持平,测试环境数据明显优化及提升
        \end{itemize}
    \end{rExperience}
\end{rSection}

% TODO:做更好的项目
% TODO:缩进
% TODO:根据公司切换项目
\begin{rSection}{项目经历}
    % \begin{rProject}{个人项目}{Minecraft 服务器}{2020.10-今}
    %     \textbf{项目概述}:开发兼容Minecraft客户端的高性能服务器后台\\
    %     \textbf{开发原因}:现有的Minecraft服务器单机支持的客户端数量较低,因此希望开发一款支持高在线人数的服务器\\
    %     \textbf{项目开发}:使用Scala开发,除去协议部分,完全自己重写,正在开发
    % \end{rProject}

    \begin{rProject}{合作项目}{WP-Reliable MD}{2018.09-2019.10}
        \textbf{项目概述}:由于使用的Markdown插件的原作者停止开发,将Markdown编辑器Tui-Editor成功引入WordPress插件商店,命名为WP Reliable-MD\\
        \textbf{项目职责}
        \begin{itemize}
            \itemsep -0.5em \vspace{-0.5em}
            \item 项目管理:担任组长,带领3开发团队,完成整体架构设计、技术选型和早期开发
            \item 项目开发:使用JavaScript,针对WordPress的界面风格设计了GUI,并替换掉编辑器的渲染引擎,重写渲染模块,使其支持Tex数学公式。采用非侵入式的方法,避免造成文章内容丢失和内容不一致。
        \end{itemize}
        \textbf{项目成果}:多人采用WP-Reliable MD作为博客的编辑器
    \end{rProject}

    \begin{rProject}{个人项目}{Escape射击游戏}{2020.02}
        \textbf{项目概述}:作为AI的训练环境和体验游戏,2D鸟瞻视角射击游戏项目\\
        \textbf{游戏设计}:该游戏需要大量AI互动,需要一定的物理特效和性能需求\\
        \textbf{项目开发}:游戏主打功能性,用C++17开发,采用Entt ECS框架,Box2D作为物理引擎,LUA作为脚本引擎,使用模板和多态开发序列化库,极大提升了C++开发能力
    \end{rProject}

    % \begin{rProject}{个人项目}{英语文章阅读器}{2019.04}
    %     \textbf{项目概述}:帮助非英语母语在网站上阅读英语文章时,查阅单词,生成背诵表而开发的网页端应用\\
    %     \textbf{功能设计}:主要功能包含全文翻译,选词翻译和自动查询功能,可以导出单词表和标记掌握的词汇\\
    %     \textbf{项目开发}:使用Python Flask作为后端,前端采用Semantic UI和jQuery
    % \end{rProject}

    % \begin{rProject}{个人项目}{卓信社交平台}{2018.08}
    %     \textbf{项目概述}:仿照Facebook和Penpal World开发的简单的社交平台\\
    %     \textbf{功能设计}:用户注册、登录、关联QQ、发帖、关注、个人资料编辑、广场、搜索用户、后台\\
    %     \textbf{项目开发}:采用的技术有PHP,HTML,JavaScript,MySQL,Semantic UI
    % \end{rProject}
\end{rSection}

\begin{rSection}{获奖经历 \& 资格认证}
    \begin{itemize}
        \itemsep -0.5em \vspace{-0.5em}
        \item RoboCom机器人大赛全球锦标赛~一等奖 \hfill 2018.07
        \item 清华大学登峰杯数据挖掘竞赛~二等奖 \hfill 2018.07 
        \item 第19届中国移动“和”教育杯电脑制作大赛~一等奖(项目第一名) \hfill 2016.07
        \item 托福英语考试~总分106(听力29,阅读30,口语21,写作26) \hfill 2019.10
        \item 中国计算机学会~认证学生会员 \hfill 2019.07
        \item 全国计算机等级考试~四级网络工程师 \hfill 2016.11
    \end{itemize}
\end{rSection}
\end{document}
