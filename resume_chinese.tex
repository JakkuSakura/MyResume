\documentclass[UTF8]{resume}

\name{邱江坤}
\address{\faPhone~+86 158 6659 6952 \faEnvelope~\href{mailto://qjk2001@gmail.com}{qjk2001@gmail.com} \faGithub~\href{https://github.com/qiujiangkun/}{GitHub: qiujiangkun} \faWeixin~JakkuSakura~~~~~}
\address{}

\begin{document}

\begin{tikzpicture}[remember picture, overlay] 
    \node[anchor = north east] at ($(current page.north east)+(-1.3cm,-1.2cm)$) {\includegraphics[height=2.5cm]{avatar.jpg}};
\end{tikzpicture}
  
\begin{rSection}{\faCogs~编程经验}
    \begin{itemize}
        \itemsep -0.5em
        \item 自小学5年级起开始学习编程,热衷开源,能够独自或合作编写上万行代码的项目,半工半读
        \item 熟悉Rust、C++、Java、Python、Scala、Golang, Solidity; 了解JavaScript、Bash、SQL、C\#、PHP、汇编
        \item 熟悉Linux、git、web3、机器学习、深度学习;了解Kafka、K8S、Netty、PostgreSQL、Redis、Disruptor、并发编程、JVM调优、编译原理
    \end{itemize}
    
\end{rSection}
\begin{rSection}{\faGraduationCap~教育经历}
    \begin{itemize}
        \item 香港科技大学~计算机工程专业\hfill 2020.9-今
    \end{itemize}
\end{rSection}
 % 跟单交易
\begin{rSection}{\faBriefcase~工作经历}
    \begin{rExperience}{Market Access}{Qube Research and Technologies}{2023.02-2023.08}
        \begin{itemize}
            \itemsep -0.5em \vspace{-0.5em}
            \item 使用C++实现和优化高性能的链接,用于处理市场数据和执行交易订单。
            \item 开发了一个全面的框架和高级工具,能够高效地管理和转换大量的历史市场数据。
            \item 创建了图形化分析工具,用于评估和改进订单管理系统的性能,为决策和优化提供有价值的见解。
            \item 设计和开发了针对各个交易所的模拟器,能够在模拟的实时环境中进行准确的交易策略测试和评估。
            \item 学习了用于回测的数学技巧
        \end{itemize}
    \end{rExperience}
    \begin{rExperience}{实习~抖音服务架构}{北京字节跳动技术有限公司}{2022.04-2022.09}
        \begin{itemize}
            \itemsep -0.5em \vspace{-0.5em}
            \item 独立提出并实现\href{https://github.com/qiujiangkun/git-fuse}{git-fuse},此工具完美解决在codegen情形下,大仓库拉取时间过长、存储空间过大问题,将总处理时间缩短到1/20,磁盘占用减少到1/100,用于Overpass等服务上,并就此进行部门内技术分析,听众达百人
            \item 结合公司内部需求,编写通用API Linter,同时支持protobuf和thrift格式,支持自定义规则和在线lint,可以作为CLI和CI程序运行
            \item 维护改进DevFlow,支持更多业务线,一个结合在线编辑、code review、代码lint、兼容性检查、代码生成、测试,方便修改API定义文件的内部平台
            \item 维护改进Overpass,一个公司内广泛使用(1.1万项目),收集RPC定义文件并生成go代码的平台
        \end{itemize}
    \end{rExperience}
    \begin{rExperience}{全职 后端开发}{ERX Inc, Thailand}{2022.03-今}
        \textbf{项目背景}:
        重写ERX.io,泰国唯一政府许可的数字资产交易所\\
        \textbf{项目内容}
        \begin{itemize}
            \itemsep -0.5em \vspace{-0.5em}
            \item 用SmartPy改写Haskell Morley的Tezos合约
            \item 从0重构交易所,着重监管合约,参与架构设计,数据库设计,API设计和开发,撮合引擎原型研发和数据导入
            \item 研究加密算法和HSM,使用特定密码学硬件进行crypto的签名操作
        \end{itemize}
    \end{rExperience}
    \begin{rExperience}{全职 后端开发}{Infinity Force Inc, Singapore}{2022.02-2022.04}
        \textbf{项目背景}:
        围绕 Axie Infinity 游戏打造工会系统\\
        \textbf{项目内容}
        \begin{itemize}
            \itemsep -0.5em \vspace{-0.5em}
            \item 用Solidity实现Infinity Force Token,即将上线
            \item 利用Rust优化现有Python+Postgres的后端开发框架,利用Code Generation优化workflow
            \item 挖掘游戏私有API,使用web3实现工会资产管理
        \end{itemize}
    \end{rExperience}
    \begin{rExperience}{软件开发实习生}{麦穗人工智能~上海穰川信息技术有限公司}{2020.12-2021.04}
        \textbf{实习内容}
        \begin{itemize}
            \itemsep -0.5em \vspace{-0.5em}
            \item 搭建分布式限流器:利用Akka和Redis和经过修改过的滑动日志法,搭建分布式限流器,针对每个用户和每个API限速,解决了经过Ingress负载均衡后各节点上Akka Stream无法将流量限制到较小的数目的问题
            % \item 对接企业微信应用:用Python开发企业微信第三方应用原型,打通智能招聘的添加联系方式和发放招聘资料环节
            \item 低速刷新服务:从PostgreSQL读取变更日志,清洗出每个租户的变更数据,根据每个租户限速配置,写入不同Kafka topics;实现供运维使用的低速刷新任务队列和Web控制台,以受限SQL注入的形式配置低速刷新任务队列
            % \item 地理位置匹配服务:使用Rust WASM重写位置匹配服务,实现trie数据结构,部署在AWS Lambda上
            % \item 重建测试环境:由于技术升级,将旧版本的基于Play Framework的测试样例迁移到到ZIO Test
        \end{itemize}
    \end{rExperience}
    \begin{rExperience}{兼职 核心开发人员}{数字货币高频交易 私人项目}{2020.07-2022.02}
        \textbf{项目背景}:
        升级现有客户端并发网络库,降低交易网站信息收集处理数据延迟,编写高频交易的底层库\\
        \textbf{项目内容}
        \begin{itemize}
            \itemsep -0.5em \vspace{-0.5em}
            \item 搭建基础设施:使用Rust语言,运用缓存行优化和Ring Buffer等底层优化,针对项目任务特点实现异步runtime,性能超过tokio/glommio/mononio; 利用Core Affinity优化性能;基于UDP实现转发器,绕开服务器机房对于IP的限制;同时支持Arm架构和x86架构,Linux和MacOS系统;基于DPDK和smoltcp实现和优化自己的网络协议栈,有效利用多核心多网卡;参考Akka Actor实现Rust上的单机Actor系统,大幅提高系统并发性能;实现跨语言数据结构和转义工具unidef,强于json-typedef-codegen; 从开源js代码库ccxt进行代码分析,完善自己系统。
            \item 搭建监控系统:运用Prometheus和Grafana搭建监控系统,实时监控不同交易平台数据采集处理情况
            % \item 搭建绘图引擎:尝试Python下多种绘图引擎,最终采用利用Dash Plotly搭建Web端绘图控制平台,前端采用React JS,后端使用Flask
            \item 搭建分布式回测系统:低延迟从不同的数字货币交易平台收集并处理数据,将数据输出到Kafka,用Ansible实现AWS上自动扩容和派发任务,用PostgreSQL+TimescaleDB收集分析回测结果;实现Kafka在AWS S3上的部署
            \item 实现JSON解析库:用过程宏、CPU缓存、减少缺页错误,将JSON反序列化效率提高到8倍,快于serde库一倍 
        \end{itemize}
    \end{rExperience}
\end{rSection}

% TODO:做更好的项目
% TODO:根据公司切换项目
\begin{rSection}{\faUsers~项目经历}

    \begin{rProject}{校内科研}{SHLL 编程语言}{2022.01-今}
        \textbf{项目概述}:SHLL是研究高层次编译器优化的编程语言,希望用于高频交易软件的编写,源自于参与的校内难以维护的相似项目,但有更多工程上的考量。\\
        \textbf{项目开发}:构建一个便于使用,美观的通用语言,通过广泛分析和使用特化以及内联,优化掉抽象代价,确保真正的零代价优化。使用Scala 3作为前端语言,保证语言的表达能力。
    \end{rProject}
    \begin{rProject}{个人项目}{Unidef}{2021.07-今}
        \textbf{项目概述}:Unidef是一个通用的类型转换,代码生成,转译的框架,用Python开发,后用Scala重写\\
        \textbf{项目开发}:维护一套统一的跨语言的类型模型和AST,力求保留类型尽可能多的语义,同时不失一般性。支持从json,jsonschema,FIX,各大语言之间类型的互相转换和代码转译。也作为高层次编译器优化(SHLL)的实验场地
    \end{rProject}

    % \begin{rProject}{个人项目}{Minecraft 服务器}{2020.10-今}
    %     \textbf{项目概述}:开发兼容Minecraft客户端的高性能服务器后台\\
    %     \textbf{开发原因}:现有的Minecraft服务器单机支持的客户端数量较低,因此希望开发一款支持高在线人数的服务器\\
    %     \textbf{项目开发}:使用Scala开发,除去协议部分,完全自己重写,正在开发
    % \end{rProject}
    % \begin{rProject}{个人项目}{Kafcat──Kafka命令行工具}{2021.3-今}
    %     \textbf{项目概述}:现有工具kafkacat的功能不够强大,提交Pull Request后发现代码难以维护,也不在积极开发,于是新建立项目,采用新语言和架构开发\\
    %     \textbf{项目开发}:使用Rust开发,完全使用异步编程,支持librdkafka和rust-kafka两个实现,支持topic之间复制,topic导出导出为json,利用管道使针对标准输入输出的程序使用Kafka
    % \end{rProject}

    % \begin{rProject}{合作项目}{WP-Reliable MD}{2018.09-2019.10}
    %     \textbf{项目概述}:原自用Wordpress Markdown插件的原作者停止开发,于是将Markdown编辑器Tui-Editor成功引入WordPress插件商店\\
    %     \textbf{项目开发}:使用JavaScript,针对WordPress的界面风格设计了GUI,并替换掉编辑器的渲染引擎,重写渲染模块,使其支持Tex数学公式。采用非侵入式的方法,避免造成文章内容丢失和内容不一致。
    % \end{rProject}
  
    % \begin{rProject}{个人项目}{Escape射击游戏}{2020.02}
    %     \textbf{项目概述}:作为AI的训练环境和体验游戏,2D鸟瞻视角射击游戏项目\\
    %     \textbf{游戏设计}:该游戏需要大量AI互动,需要一定的物理特效和性能需求\\
    %     \textbf{项目开发}:游戏主打功能性,用C++17开发,采用Entt ECS框架,Box2D作为物理引擎,LUA作为脚本引擎,使用模板和多态开发序列化库,极大提升了C++开发能力
    % \end{rProject}

    % \begin{rProject}{个人项目}{英语文章阅读器}{2019.04}
    %     \textbf{项目概述}:帮助非英语母语在网站上阅读英语文章时,查阅单词,生成背诵表而开发的网页端应用\\
    %     \textbf{功能设计}:主要功能包含全文翻译,选词翻译和自动查询功能,可以导出单词表和标记掌握的词汇\\
    %     \textbf{项目开发}:使用Python Flask作为后端,前端采用Semantic UI和jQuery
    % \end{rProject}

    % \begin{rProject}{个人项目}{卓信社交平台}{2018.08}
    %     \textbf{项目概述}:仿照Facebook和Penpal World开发的简单的社交平台\\
    %     \textbf{功能设计}:用户注册、登录、关联QQ、发帖、关注、个人资料编辑、广场、搜索用户、后台\\
    %     \textbf{项目开发}:采用的技术有PHP,HTML,JavaScript,MySQL,Semantic UI
    % \end{rProject}
\end{rSection}

\begin{rSection}{\faAward~获奖经历~资格认证}
    \begin{itemize}
        \itemsep -0.5em
        \item ByteCamp 2021 自动代码缺陷修复方向 软件工程小组第三名 \hfill 2021.08
        \item 托福英语考试~总分106(听力29,阅读30,口语21,写作26) \hfill 2019.10
        \item 信息学奥赛(NOIP)提高组山东赛区一等奖 \hfill 2017.11/2018.11
        % \item 中国计算机学会~认证学生会员 \hfill 2019.07
        \item RoboCom机器人大赛全球锦标赛~一等奖 \hfill 2018.07
        % \item 清华大学登峰杯数据挖掘竞赛~二等奖 \hfill 2018.07 
        \item 全国计算机等级考试~四级网络工程师 \hfill 2016.11
        % \item 第19届中国移动“和”教育杯电脑制作大赛~一等奖(项目第一名) \hfill 2016.07
    \end{itemize}
\end{rSection}
\end{document}
